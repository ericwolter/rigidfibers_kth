\chapter{Numerical algorithm and serial implementation}
\label{cha:serial_implementation}

In the last chapter, we presented the theoretical foundation of the physics and mathematics involved in simulating rigid fibers. Based on the Stokes Equation, we introduced the framework of boundary integral methods to efficiently model the behavior of rigid fibers. Using this background we will now review the numerical approach used for the simulation.

We will separate the overall algorithm into four steps and discuss each step individually. Additionally, we will touch upon implementation details used in the original serial version. We close the chapter with a brief reflection of the performance characteristics of the serial implementation to guide the parallel implementation on the GPU.

\section{Assemble System}
In the previous chapter we obtained the final closed system and the equations~\eqref{eq:velocity_centerline}~and~\eqref{eq:slender_boundary_constraints}. By integrating equation~\eqref{eq:velocity_centerline} from $-1$ to $1$ we can derive two separate equations for $\mathbf{\dot{x}}_m$ and $\mathbf{\dot{t}_m}$. Using these two equations and some algebra we can obtain a system of equations for $\mathbf{f}_m$ for all fibers from $m=1,2,\dots,M$, which only include computable quantities. For more details and an in depth discussion please refer to the original paper by Tornberg and Gustavsson~\cite{Tornberg2006}.

In order to be able to solve these equations for $\mathbf{f}_m$ we have to discretize them. For this we expand the force as a sum of $N+1$ Legendre polynomials $P_n(s)$
\begin{equation}
  \label{eq:force_discretization}
  \mathbf{f}_m = \frac{1}{2}\mathbf{F}_g + \sum_{n=1}^{N}a_{m}^{n} P_n(s) \text{,}
\end{equation}
where the coefficients $a_{m}^{n}$ are vectors with three components. The number of Legendre polynomials $N$ used for the force expansion is a parameter and is set to $5$ in our simulation.

\subsection{Closed linear system}
\label{subsec:closed_system}

Using the force expansion from equation~\ref{eq:force_discretization} in equation~\ref{eq:slender_fibers_velocity} we get a closed linear system of equations for the coefficients $a_{m}^{n}$ for $n=1,\dots,N$ force expansions and $m = 1,\dots,M$ fibers.

For a 3-dimensional simulation, this results in a linear system of size $3MN\times3MN$. The first step of the algorithm is thus to compute and assemble the linear system in memory. Writing the system in the standard form $\mathbf{A}\mathbf{\bar{a}}=\mathbf{b}$ gives the following structure for the matrix $A$ and the right-hand side $b$
\begin{equation}
  \label{eq:matrix_structure}
  \renewcommand\arraystretch{1.5}
  \mathbf{A} =
  \begin{bmatrix}
    \mathbf{I} & \bar{A}_{12} & \cdots & \bar{A}_{1M} \\
    \bar{A}_{21} & \mathbf{I} & \cdots & \bar{A}_{2M} \\
    \vdots & \vdots & \ddots & \vdots \\
    \bar{A}_{M1} & \bar{A}_{M2} & \cdots & \mathbf{I}
  \end{bmatrix} \text{,} \quad \mathbf{b} =
  \begin{bmatrix}
    \bar{b}_{1} \\
    \bar{b}_{2} \\
    \vdots \\
    \bar{b}_{M} \\
  \end{bmatrix} \text{.}
\end{equation}
In this notation $\bar{A}_{ml}$ describes the $3N\times3N$ matrix encapsulating the contribution from the force coefficients on fiber $l$ onto the force coefficients for fiber $m$.

\subsection{Inner integral}
\label{subsec:inner_integral}

For each $\bar{A}_{ml}$, a $3\times3$ matrix, $\Theta_{lm}^{kn}$, where
\begin{equation}
  \label{eq:inner_integral}
  \Theta_{lm}^{kn} = \int_{-1}^{1} \left[\int_{-1}^{1}\mathbf{G}(\mathbf{R}(s,s')) P_k(s') \, ds' \right]P_n(s) \, ds \text{,}
\end{equation}
has to be evaluated for each force index $k,n = 1,2,\dots,N$. $G$ and $R$ are defined as in equation~\eqref{eq:G}~and~\eqref{eq:fiber_distance}, respectively. This can be achieved by using a standard Gaussian quadrature approach for both the inner and outer integral. However, as fibers get very close the number of quadrature points have to increase to accurately compute the fiber interactions. Unfortunately, doing so also increases the unknowns in the system, which in turn decreases the performance. For our simulation we use the same approach as the original paper. We divide each fiber into 8 subintervals and use a three-point gaussian quadrature on each interval. This results in a total of $3 \times 8 = 24$ quadrature points per fiber, which represents a good trade-off between accuracy and performance.

Another option is to find an analytical solution for the inner integral and only solve the outer integral numerically. We will not discuss the detailed derivation of the analytical solution, for an in-depth discussion please see~\cite{Tornberg2006}. In theory this approach allows for perfect accuracy for the inner integral, however in practice this is limited by the numerical precision of the simulation. The obtained formulas are recursive and thus sensitive to round off errors. To minimize the accumulation of round off errors, the original serial implementation uses a trick and switches the direction of the recursion, depending on how far apart the fibers are. This improves the practical accuracy and did not have a negative effect on performance.

Choosing between both options requires a careful examination of the accuracy and performance trade-off and is depend on the simulation setup. For the original serial implementation the combined numerical and analytical approach for evaluation proved to be the fastest and was thus chosen as the default. We will later explore how is applies to the new parallel GPU implementation.

\section{Solve System}

After having assembled the linear system $\mathbf{A}\mathbf{\bar{a}}=\mathbf{b}$ the next step to solve it. This can be done using standard linear equation solvers. This step is treated as a black box by the simulation, simply plugging in the matrix $\mathbf{A}$ and the vector $\mathbf{b}$ and getting back the solution vector $\mathbf{\bar{a}}$ representing the force coefficients.

The linear system can either be solved using a direct solver or an iterative method like GMRES. Unfortunately the matrix is neither symmetric nor sparse so we can't take advantage of specialized solvers for these cases~\cite{Tornberg2006}. As long as the fibers are sufficiently far apart and the matrix is not ill-conditioned, GMRES is able to solve the system in less than $10$ iterations. This is the reason why the original serial implementation uses GMRES by default. How the different solvers perform on the GPU will be compared in section~\ref{sec:bench_linear_solvers}.

\section{Update Velocities}

The force coefficients obtained by solving the linear system can now be used to calculate the remaining unknowns. By using the force expansion in the two separate equations for $\mathbf{\dot{x}}_m$ and $\mathbf{\dot{t}_m}$. The required implementation is thus similar to the computations for the \emph{Assemble System} step. The two integrals, again, can be evaluated either exclusively numerical or with the combined approach of solving the inner integral analytically.

\section{Update Fibers}
\label{sec:serial_update_fibers}

The final step takes care of advancing the fibers forward in time. The equations do not impose any strict stability restrictions, so an explicit time-stepping scheme can be used. We also use the same second order multi-step method used in the original paper. The update for the position of the center coordinate $\mathbf{x}_m$ is given by the following discretization in time
\begin{equation}
  \label{eq:time_discretization}
  \frac{3\mathbf{x}_m^{i+1} - 4\mathbf{x}_m^{i} + \mathbf{x}_m^{i-1}}{2 \Delta t} = (2\mathbf{\dot{x}}_m^{i} - \mathbf{\dot{x}}_m^{i-1}) \text{,}
\end{equation}
where the time step is denoted by $\Delta t$ and superscripts denote the numerical approximation of $\mathbf{x}_m(t_i)$. In order to compute the next state this method requires both the previous and the current state. As there is no previous time step for $t_0$, the first step $\mathbf{x}_{m}^{1}$ is computed by a simply first order forward Euler method. The calculation for the orientation vector $\mathbf{t}_m$ is computed using the same discretization by replacing the translational velocity $\mathbf{\dot{x}}_m$ with the rotational velocity $\mathbf{\dot{t}}_m$. Additionally, we must renormalize the orientation vector so that it maintains its unit length.

At the end of this step the state of the fibers can optionally be written to an external file for post processing and visualization in other tools. After completing this step, the algorithm starts again from the top with the \emph{Assemble System} step. This cycle repeats until a specified number of time steps have been executed.

\section{Algorithm summary}
\label{sec:algorithm_summary}

The original paper implemented this algorithm using Fortran. The computation were all performed in double precision and executed using a single thread on the CPU. In summary the four steps of the algorithm are at each timestep $t=t^n$ given the fiber configuration in terms of position, $x_m$, and orientation, $t_m$, 
\begin{description}
  \item[1. Assemble System] Computes all interactions between fibers and yields linear system $\mathbf{A}\mathbf{\bar{a}}=\mathbf{b}$
  \item[2. Solve System] 
  \item[3. Update Velocities] Computes velocities $\mathbf{\dot{x}}$ and $\mathbf{\dot{t}}$
  \item[4. Update Fibers] Yields new fiber configuration at $t=t^{n+1}$
\end{description}

Empirical results show that the majority of the required computation time is spent on the 1.~\emph{Assemble System} and 3.~\emph{Update Velocities} steps. The time required for advancing the simulation state in step 4.~\emph{Update Fibers} is completely negligible. We will see later in Chapter~\ref{cha:benchmarks} that the same holds true for the new parallel implementation.

Since we won't be writing our own implementation of linear solvers on the GPU we have only limited influence on the performance of the 2.~\emph{Solve System} step. We instead treat it as a black box and rely on the efficiency of pre-existing libraries. The only control we have is the choice which library and solver to use. For direct solvers the computation time only depends on the number of unknowns and is approximately constant throughout the simulation. The performance of iterative solvers on the other hand is highly depend on the condition number of the matrix and thus unpredictable. In line with the original paper we will both test a direct solver and iterative solvers to get a better understanding of their respective performance behavior.

Consequently, the most important step to optimize is 1.~\emph{Assemble System}. Fortunately it is well suited for parallelization. The fibers can be partitioned naturally across the compute units, where each unit is responsible for a subset of fibers. The focus of the optimization will thus lie on the \emph{Assemble System}. As the required computation for the 3.~\emph{Update Velocities} step is similar it will also indirectly benefit from any optimizations. Additionally, we will also look at how the two different options for solving the integral in equation~\ref{eq:inner_integral} perform in the parallel environment. Therefore, the next goal is to parallelize each algorithm step on the GPU.
