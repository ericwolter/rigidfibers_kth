\chapter{Simulation Parameters}
\label{app:parameters}

\begin{table}[ht]
  \caption*{Parameters for GMRES iterations experiment in Sec~\ref{subsec:example_concentration_gmres}.}
  \begin{center}
    \begin{tabulary}{\textwidth}{LR}
      \toprule
      Parameter & Value \\
      \midrule
      Number of fibers & $2000$ \\
      Average distance & $0.02–40.96$ \\
      Timestep & $0.1$ \\
      Slenderness & $0.01$ \\
      Number of terms in force expansion & $5$ \\
      Number of quadrature intervals & $8$ \\
      Number of quadrature points per interval & $3$ \\
      \bottomrule
    \end{tabulary}
  \end{center}
\end{table}

\begin{table}[ht]
  \caption*{Parameters for tumbling orbits experiment in Sec~\ref{sec:example_ring}.}
  \begin{center}
    \begin{tabulary}{\textwidth}{LR}
      \toprule
      Parameter & Value \\
      \midrule
      Number of fibers & $16$ \\
      Average distance & $0.2146$ \\
      Timestep & $0.1$ \\
      Slenderness & $0.01$ \\
      Number of terms in force expansion & $5$ \\
      Number of quadrature intervals & $8$ \\
      Number of quadrature points per interval & $3$ \\
      \bottomrule
    \end{tabulary}
  \end{center}
\end{table}

\begin{table}[ht]
  \caption*{Parameters for sedimenting sphere experiment in Sec~\ref{sec:example_sphere}.}
  \begin{center}
    \begin{tabulary}{\textwidth}{LR}
      \toprule
      Parameter & Value \\
      \midrule
      Number of fibers & $2000$ \\
      Average distance & $0.2$ \\
      Timestep & $0.1$ \\
      Slenderness & $0.01$ \\
      Number of terms in force expansion & $5$ \\
      Number of quadrature intervals & $8$ \\
      Number of quadrature points per interval & $3$ \\
      \bottomrule
    \end{tabulary}
  \end{center}
\end{table}

\begin{table}[ht]
  \caption*{Parameters for fiber concentration effect on break-up in Sec~\ref{subsec:effect_concentration}.}
  \begin{center}
    \begin{tabulary}{\textwidth}{LR}
      \toprule
      Parameter & Value \\
      \midrule
      Number of fibers & $2000$ \\
      Average distance & $0.08–2.56$ \\
      Timestep & $0.1$ \\
      Slenderness & $0.01$ \\
      Number of terms in force expansion & $5$ \\
      Number of quadrature intervals & $8$ \\
      Number of quadrature points per interval & $3$ \\
      \bottomrule
    \end{tabulary}
  \end{center}
\end{table}

\begin{table}[ht]
  \caption*{Parameters for number of fibers effect on break-up in Sec~\ref{subsec:effect_number}.}
  \begin{center}
    \begin{tabulary}{\textwidth}{LR}
      \toprule
      Parameter & Value \\
      \midrule
      Number of fibers & $100–2000$ \\
      Average Distance & $0.4$ \\
      Timestep & $0.1$ \\
      Slenderness & $0.01$ \\
      Number of terms in force expansion & $5$ \\
      Number of quadrature intervals & $8$ \\
      Number of quadrature points per interval & $3$ \\
      \bottomrule
    \end{tabulary}
  \end{center}
\end{table}

\begin{table}[ht]
  \caption*{Parameters for un/mixed spherical cloud in Sec~\ref{sec:mixed_density_sphere}.}
  \begin{center}
    \begin{tabulary}{\textwidth}{LR}
      \toprule
      Parameter & Value \\
      \midrule
      Number of fibers & $2000$ \\
      Average Distance & $0.4$ \\
      Timestep & $0.1$ \\
      Slenderness & $0.01$ \\
      Number of terms in force expansion & $5$ \\
      Number of quadrature intervals & $8$ \\
      Number of quadrature points per interval & $3$ \\
      \bottomrule
    \end{tabulary}
  \end{center}
% hack to get the figure to the top of the page
% see:http://stackoverflow.com/questions/2009420/how-to-put-a-figure-on-the-top-of-a-page-on-its-own-in-latex
  \vspace*{6in}
\end{table}
