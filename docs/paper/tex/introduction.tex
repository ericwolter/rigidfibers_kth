\chapter{Introduction}

Predicting the physical behavior of particles suspended in fluids is of great interest in a variety of different fields. The ability to accurately model and simulate different categories of particle suspensions allows their properties to be analyzed and optimized for a large number of applications. Examples include medical applications were the delivery and distribution of the active agent has to be modeled, waste management to efficiently extract waste from water and the paper industry which is trying to improve the characteristics of their material.

Even in the simplest flow cases, the flow of a particle suspension exhibits very complex and complicated dynamical behavior. The rheological properties of the suspension depend strongly on features such as the concentration of particles, particle shapes and particle interactions. In order to accurately capture the complex dynamics of a suspension using numerical simulations, a large number of particles are required in the simulation. Hence, the ability of the numerical algorithm to efficiently handle a large amount of particles is of crucial importance.

The work in this thesis focuses on increasing and optimizing the efficiency of a numerical algorithm of the simulation of sedimenting rigid and slender fibers in a viscous fluid. The fibers are modeled and simulated on a particle-level in a 3D free-space Stokes fluid. The mathematical model is based on a boundary integral formulation and a non-local slender body approximation by Tornberg and Gustavsson~\cite{Tornberg2006}.

There are many numerical studies of fiber suspension and several different methods have been developed for both rigid and flexible fibers. One approach is the so-called beads-model, where the fibers are modeled as a set of connected spherical beads (e.g.~\cite{Fan1998}\cite{Joung2001}\cite{Skjetne1997}\cite{Yamamoto1995}). The immersed boundary method discretizes the fibers with Lagrangian markers and distributes the force onto a background grid which is then used to modify the fluid flow (e.g.~\cite{Peskin2002}\cite{Stockie1998}). The last approach is based on slender body theory which uses the large aspect ratio of the fibers to simplify the underlying model (e.g.~\cite{Gustavsson2009}\cite{Tornberg2006}\cite{Tornberg2004}). A comprehensive review of the numerical studies of fiber suspension can be found in Guazzelli and Hinch~\cite{Guazzelli2011}.

In this thesis we develop a high performance GPU implementation of the numerical algorithm for simulating fiber suspension. By taking advantage of the massively parallel architecture of modern GPUs many more fibers can be simulated compared to the previous CPU based implementation of the algorithm. The major goal is to easily and efficiently perform simulations on a high-end desktop computer or workstation readily accessible by the researcher. This will allow the researcher to explore the huge problem space and simulate a large number of fibers in a short amount of time. Thus, it enhances the capacity to rapidly iterate and discover interesting test cases. These cases can then be used as a starting point for large scale simulations using computing clusters.

The most costly part of the numerical algorithm is to account for the interactions between all fibers in the system. For the calculations we chose the naive algorithm, where the interactions between every pair of fibers is computed. The development of such a naive algorithm is both easier and more cost effective than an alternative fast summation approach. A fast summation approach, like the fast multipole method, is both complex to implement and introduces a potentially large performance overhead. Therefore it might not result in the desired performance increase compared to the naive algorithm. Since the interactions between every pair of fibers is computed, the naive algorithm is more accurate. Our choice of consumer-grade GPU hardware limits the simulation to single precision, thus maintaining the accuracy of is very important. By using the naive algorithm on the GPU the goal is to strike a good balance.

This thesis is divided into the three major parts: \emph{Previous Work}, \emph{GPU Implementation} and \emph{Results}. First we introduce the theoretical foundation of the numerical method in Chapter~\ref{cha:theoretical_foundation} and refer to the original paper by Tornberg and Gustavsson~\cite{Tornberg2006} for in-depth details. The numerical algorithm and the serial implementation on the CPU developed in the original paper~\cite{Tornberg2006} is presented in Chapter~\ref{cha:serial_implementation}. Together these two chapters form the \emph{Previous Work} part of the thesis. Based on this the \emph{GPU Implementation} part follows. Chapter~\ref{cha:gpu_programming} briefly introduces general purpose computing on the GPU. Combining the previous work and GPU computing, we then describe our new parallel implementation of the rigid fibers simulation on GPUs in Chapter~\ref{cha:parallel_implementation}. This represents the base for the efficiency and the performance improvements and therefore is the core of the thesis. Afterwards, the final \emph{Results} are presented. The benchmarks in Chapter~\ref{cha:benchmarks} illustrate the achieved performance increase of the parallel \emph{GPU Implementation}. Finally, we performed and compared our simulation results to a few known experiments in Chapter~\ref{cha:experiments}.
