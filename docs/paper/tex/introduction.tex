\chapter{Introduction}

Predicting the physical behavior of particles suspended in fluids is of great interest in a variety of different fields. The ability to accurately model and simulate different kind of particle suspensions allows their properties to be analyzed and optimized for a large number of applications. Examples include medical applications were the deliver and distribution of the active agent has to be modeled, waste management to efficiently extract waste from water and the paper industry which is trying to improve the properties of their material.

Even in the simplest flow cases, the flow of a particle suspension exhibits very complex and complicated dynamical behavior. The rheological properties of the suspension depend strongly on features such as the concentration of particles, particle shapes and particle interactions. In order to accurately capture the complex dynamics of a suspension using numerical simulations, a large number of particles are required in the simulation. Hence, the ability of the numerical algorithm to efficiently handle a large amount of particles is of crucial importance.

The work in this thesis focuses on increasing and optimizing the efficiency of a numerical algorithm of the simulation of sedimenting rigid and slender fibers in a viscous fluid. The fibers are modeled and simulated on a particle-level in a 3D free-space Stokes fluid. The mathematical model is based on a boundary integral formulation and a non-local slender body approximation by Tornberg and Gustavsson\cite{Tornberg2006}. 

There are many numerical studies of fiber suspension, see e.g. Guazzelli and Hinch\cite{Guazzelli2011} and references therein.

In this thesis we present a high performance GPU implementation of the numerical algorithm for simulating fiber suspension. By taking advantage of the massively parallel architecture of modern GPUs many more fibers can be simulated compared to the previous CPU based implementation of the algorithm. We focus the development on the ability to easily and efficiently perform simulation on a high-end desktop computer or workstation readily accessible by the researcher. This approach allows the researcher to explore the huge problem space and simulate a relatively large number of fibers in a short amount of time. Thus, it enhances the capacity to rapidly iterate and to find and discover interesting test cases. Theses cases can than be used as a starting point for large scale simulation using computing clusters.

The most costly part of the numerical algorithm is to account for the interactions between all fibers in the system. Here we limit ourselves to the naive algorithm for simulating the fibers, were the interactions between every pair is computed. The development of such a naive algorithm is not only easier and more cost effective but the potentially large overhead introduced by a complex fast summation method, like the fast multipole method, might not result in desired performance increase. Additionally, because we are limited to a single precision simulation by our choice of consumer-grade GPU hardware, maintaining the accuracy of the simulation is a major concern. However, exploring theses questions is an interesting future research direction.

This thesis is divided into three major parts. First we introduce the theoretical foundation of the numerical method in Chapter~\ref{cha:theoretical_foundation} and refer to the original paper by Tornberg and Gustavsson\cite{Tornberg2006} for in-depth details. We conclude the description of the previous work by discussing the numerical algorithm and the serial implementation on the CPU in Chapter~\ref{cha:serial_implementation}. In the following Chapter~\ref{cha:gpu_programming} we briefly introduce the topic of general purpose computing on GPUs. Based on this we then describe our new parallel GPU implementation of the rigid fibers simulation in detail in Chapter~\ref{cha:parallel_implementation}. The last part of this thesis presents the benchmarking result and achieved performance increase of the GPU implementation in Chapter~\ref{cha:benchmarks}. Finally, we performed and compared our simulation results to a few known experiments in Chapter~\ref{cha:experiments}.
